\documentclass[11pt]{article}

\usepackage[utf8]{inputenc}
\usepackage[margin=1in]{geometry}
\usepackage{microtype}
\usepackage{fancyhdr}
\usepackage{palatino}
 
\pagestyle{fancy}
\fancyhf{}
\lhead{Nathaniel T. Stemen}
\rhead{University of Oregon}

\lfoot{Statement of Purpose}
\rfoot{Page \thepage}

\renewcommand{\baselinestretch}{1.27}

\begin{document}

Tricks! ``Tricks'' have always inspired my passion and relentless drive for solutions and mastery. Before college, I devoted my free time to conquering the physical skills and knowledge required for high-level skateboard tricks. Silly? Not if you consider that I was unwittingly training myself in keen observation, physical dynamics, intense mental focus, hard work on fundamentals, imaginative application of available tools, teamwork, and personal initiative. Toward the end of high school and at NYU, my interests in seemingly mysterious physical phenomena evolved into an immersion in mathematics.

My passion for mathematics deepened during my research at Yale University where we searched for a hypothetical sterile neutrino. There I optimized our data collection pipeline using pulse shape discrimination and built a simulation of the detector that incorporated realistic photomultiplier tubes and quantum efficiency curves. Using my simulation we were able to design a light-guide that collected 30\% more photons and increased detector uniformity. In between writing code and taking samples in the lab, I began exploring the mathematics behind neutrino oscillation which led me to the world of linear algebra. I learned about mixing matrices and angles, but also more foundational material about unitary transformations and operator spectra. As I wrapped up my time at Yale, I presented a poster at a large conference and published two papers with with the collaboration.

In my third year, I took a reading course on differential geometry which was solely discussion based. Embracing this format, I was stunned how easily direct, imaginative, and new discovery became. I learned about classical differential geometry, and even built a cardboard and string model of a hyperboloid to demonstrate how it was doubly-ruled. We also dove into topics like manifolds, the Gauss-Bonnet theorem, differential topology and surgery theory. Bettering my presentation skills through weekly reports to my peers plays into my ultimate goal of collaborating on research teams and teaching at the university level.

% The following semester was difficult for me: a more painful learning experience. After slipping behind academically, and losing my grandfather, I ended this semester with my lowest course grade ever. The semester was not a total disaster though, because the difficult courses forced me out of my comfort zone. \textit{Thermodynamics and Statistical Physics} showed me I wasn't used to thinking in combinatorial terms, and \textit{Topology} pushed me to understand deeper facts about local and global properties such as connectedness. I live, and believe in, the principle that failure often presents an entry point to a better path toward solution and success.

Working with Prof. Luciano Medina on optical vortex solitons, our mission was the prove the existence of solutions for a class of nonlinear Schr\"{o}dinger equations. Using a variational method, we reframed the problem in the language of functional analysis and operator algebras. Doing this allowed us to use powerful tools from functional analysis and apply them to a functional which we were trying to minimize. We were thus able to give necessary conditions for nontrivial solutions and prove the existence of solutions via constrained minimization. To further understand the problem, I computed solutions using python and verified the numerics from previous papers. While the work was highly specialized, it bolstered my fundamentals in functional analysis with topics like Sobolev spaces, trace operators, measure theory and lots of numerical computing as well. This work led to a poster presentation for NYU faculty, and talk at Rice University.

Although my program did not require a thesis, I chose to write one on Q-Balls (a non-topological soliton) knowing it would help prepare me for future research. Studying the foundational classical and quantum field theories which underlie much of our work allowed me to give physical explanations as well as mathematical ones for our results. Through proving integral inequalities and functionals to be lower semi-continuous and coercive, we showed spherically symmetric Q-Ball solutions existed. This experience taught me lots of technical functional analysis, and advanced field theories, but my real takeaway was the knowledge that I enjoy research that touches both pure and applied mathematics.

This has largely drawn me to fields that intertwine ideas from multiple areas of mathematics, and mathematics adjacent fields. Algebraic geometry, and representation theory are two fields of mathematics that both interest me as a pure field of study and find many applications in other fields. Both of these fields are extensively used in mathematical physics to better understand string theory, and quantum field theories. Since the University of Oregon has multiple professors working in these fields, such as Ben Elias, Benjamin J.\ Young and Yefeng Shen, I believe this to be an institution at which I can excel in such research. Along with the universities devotion to helping graduate students succeed with programs like the ``pre-school'' and the first year mentorship program, I know the transition from working to being an academic will be made much easier.

After graduating from NYU I opted to take a job as a software developer/data scientist at the academic writing enterprise; Overleaf. Having no professional software experience, I was nervous that I was in over my head. However, I demonstrated that I could pick up necessary skills on the job. My time is split between working on our web application and handling data science related tickets, but even do \LaTeX{} support. Most importantly though, this job has taught me organizational skills that will benefit me as a graduate student. 

Overleaf has also sent me to a multitude of mathematics conferences to represent our company. While I am at conferences to interact with users, I have the freedom to attend talks and lectures as I wish. This has allowed me to further explore areas I'm interested in and have attended talks ranging from linguistic diversity in the classroom, to category theory. All in all becoming a more proficient programmer has only strengthened my mathematical ability. Running simulations of shuffling cards, building computational models of dual numbers, and learning about how code relates to computational complexity theory are just a few examples of how programming has helped deepen and broaden my mathematical knowledge. Maintaining this connection has allowed me to grow my enthusiasm for the mathematics, and develop a more specialized interest in algebraic geometry and mathematical physics.

My treasured moments (in skateboarding and research) have resulted from discipline, struggle, and trust. Through research, experimentation, and eventual teaching, I believe I will pursue significant questions successfully. Determined to pursue research that is deeply mathematical, computational, and philosophical, the University of Oregon's mathematics PhD program will be excellently suited for the interdisciplinary research I hope to engage in. Given the similarities between my academic interests, the University of Oregon’s research faculty and environment, I trust this program will help me succeed in my mathematical pursuit. And I may skateboard around Eugene as well.

\end{document}