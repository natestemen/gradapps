\documentclass[11pt]{article}

\usepackage[utf8]{inputenc}
\usepackage[margin=1.25in]{geometry}
\usepackage{microtype}
\usepackage{fancyhdr}
\usepackage{palatino}
\usepackage{tipa}
\usepackage{booktabs}
\usepackage{enumitem}
 
\pagestyle{fancy}
\fancyhf{}
\lhead{Nathaniel T. Stemen}
\rhead{University of Washington}

\lfoot{Course Information}
\rfoot{Page \thepage}

\begin{document}

\section*{Ordinary differential equations}

\begin{center}
\begin{tabular}{lrc}
Course Name & Course Number & Grade Received \\ \toprule
Linear Algebra \& Differential Equations & MA-UY 2034 & A
\end{tabular}
\end{center}
\subsubsection*{Course Objectives \& Textbook}
\begin{enumerate}
    \item Linear Algebra \& Differential Equations
    
    Topics to be covered include: first-order equations including integrating factors; second-order equations including variation of parameters; series solutions; elementary numerical methods including Euler’s methods, Runge-Kutta methods, and error analysis; Laplace transforms; systems of linear equations; boundary-value problems.
    
    \textbf{Book}: A First Course in Differential Equations by Dennis Zill
\end{enumerate}

\section*{Linear algebra}

\begin{center}
\begin{tabular}{lrc}
Course Name & Course Number & Grade Received \\ \toprule
Linear Algebra \& Differential Equations & MA-UY 2034 & A \\
Advanced Linear Algebra \& Complex Variables & MA-UY 3113 & A \\
Linear Algebra I & MA-GY 7033 & A \\
Linear Algebra II & MA-GY 7043 & A
\end{tabular}
\end{center}

\subsubsection*{Course Objectives \& Textbook}
\begin{enumerate}
    \item Linear Algebra \& Differential Equations
    
    This course introduces vector concepts. Linear transformations. Matrices and Determinants. Characteristic roots and eigenfunctions.
    
    \textbf{Book}: Linear Algebra and its Applications by David C.\ Lay
    
    \item Advanced Linear Algebra \& Complex Variables
    
    The Gram-Schmidt process, inner product spaces and applications, singular value decomposition, LU decomposition, derivatives and Cauchy-Riemann equations, integrals and Cauchy integral theorem, power and Laurent series, residue theory.
    
    \textbf{Book}: Visual Complex Analysis by Tristan Needham and professor's notes.
    
    \item Linear Algebra I (graduate course)
    
    This course covers basic ideas of linear algebra: Groups, rings, fields, vector spaces, basis, dependence, independence, dimension. Relation to solving systems of linear equations and matrices. Homomorphisms, duality, inner products, adjoints and similarity.
    
    \textbf{Book}: Professor's notes, but studied with Abstract Algebra by 	David S. Dummit and Richard M. Foote.
    
    \item Linear Algebra II (graduate course)
    
    Topics covered are basic concepts of linear algebra continuing with: range, nullity, determinants and eigenvalues of matrices and linear homomorphisms, the polar decomposition and spectral properties of linear maps, orthogonality, adjointness and its applications.
    
    \textbf{Book}: Professor's notes and Finite Dimensional Vector Spaces by Paul Halmos.
\end{enumerate}

\section*{Numerical analysis, scientific computing}

\begin{center}
\begin{tabular}{lrc}
Course Name & Course Number & Grade Received \\ \toprule
Computational Physics & PH-UY 3614 & B- \\
Thesis for Bachelor of Science Degree & MA-UY 4993 & A
\end{tabular}
\end{center}

\subsubsection*{Course Objectives \& Textbook}
\begin{enumerate}
    \item Computational Physics

    An introduction to numerical methods. Solving ordinary differential equations, root finding, Fourier transforms, numerical integration, linear systems. Techniques are applied to projectile motion, oscillatory motion, planetary motion, potentials and fields, waves and quantum mechanics
    
    \textbf{Book}: Computational Physics by Nicholas Giordano and Hisao Nakanishi
    
    \item Thesis for Bachelor of Science Degree
    
    While working on my thesis I learned about topics ranging from simple root-finding procedures to numerically solving differential equations and extensively used the finite element method.
\end{enumerate}

\section*{Computer programming}

\begin{center}
\begin{tabular}{lrc}
Course Name & Course Number & Grade Received \\ \toprule
Introduction to Programming \& Problem Solving & CS 1114 & A- \\
Data Structures \& Algorithms & CS-UY 1134 & B
\end{tabular}
\end{center}

\subsubsection*{Course Objectives \& Textbook}
\begin{enumerate}
    \item Introduction to Programming \& Problem Solving

    The course covers fundamentals of computer programming and its underlying principles using the Python programming language.
    
    \item Data Structures \& Algorithms
    
    This course covers abstract data types and the implementation and use of standard data structures along with fundamental algorithms and the basics of algorithm analysis.
    
    \textbf{Book}: Introduction to Algorithms by T.\ H.\ Cormen, C.\ Stein, R.\ L.\ Rivest and C.\ E.\ Leiserson

\end{enumerate}

\section*{Partial differential equations}

\begin{center}
\begin{tabular}{lrc}
Course Name & Course Number & Grade Received \\ \toprule
Applied Partial Differential Equations & MA-UY 4413 & A- \\
Mathematical Physics & PHYS-UA 106 & A
\end{tabular}
\end{center}

\subsubsection*{Course Objectives \& Textbook}
\begin{enumerate}
    \item Applied Partial Differential Equations
    
    This course gives an overview of PDEs that occur commonly in the physical sciences with applications in heat flow, wave propagation, and fluid flow. Analytical as well as some numerical solution techniques will be covered, with a focus on applications.
    
    \textbf{Book}: Applied Partial Differential Equations by J.\ David Logan

    \item Mathematical Physics
    
    Mathematical preparation for further study in physics. Review of vectors and matrices, differential and integral operations, curvilinear coordinate systems, introduction to tensors, Dirac delta-function, complex variables, ordinary and partial differential equations, solutions to Laplace's equation.
    
    \textbf{Book}: Mathematical Methods in the Physical Sciences by Mary L.\ Boas
\end{enumerate}

\section*{Complex variables}

See description of \textbf{Advanced Linear Algebra \& Complex Variables} course in Linear Algebra section.

\section*{Probability and Statistics}

\begin{center}
\begin{tabular}{lrc}
Course Name & Course Number & Grade Received \\ \toprule
Introduction to Probability I & MA-UY 3012 & A \\
Probability II & MA-UY 3022 & A
\end{tabular}
\end{center}

\subsubsection*{Course Objectives}
\begin{enumerate}
    \item Introduction to Probability I \& Probability II
     
    Together, the courses cover axioms of mathematical probability, combinatorial analysis, binomial distribution, Poisson and normal approximation, random variables and probability distributions, generating functions, the Central Limit Theorem and Laws of Large Numbers, Markov Chains, and basic stochastic processes.
\end{enumerate}

\section*{Advanced Calculus and/or real analysis}

\begin{center}
\begin{tabular}{lrc}
Course Name & Course Number & Grade Received \\ \toprule
Honors Analysis I & MATH-UA 328 & B+ \\
Reading in Mathematics I (Differential Geometry) & MA-GY 9413 & A
\end{tabular}
\end{center}

\subsubsection*{Course Objectives \& Textbook}
\begin{enumerate}
    \item Honors Analysis I
    
    This is an introduction to the rigorous treatment of the foundations of real analysis in one variable. It is based entirely on proofs. Students are expected to know what a mathematical proof is and are also expected to be able to read a proof before taking this class. Topics include: properties of the real number system, sequences, continuous functions, topology of the real line, compactness, derivatives, the Riemann integral, sequences of functions, uniform convergence, infinite series and Fourier series. Additional topics may include: Lebesgue measure and integral on the real line, metric spaces, and analysis on metric spaces.
    
    \textbf{Book}: Foundations of Mathematical Analysis by Richard Johnsonbaugh and W.\ E.\ Pfaffenberger

    \item Differential Geometry
    
    The geometry of curves and surfaces in Euclidean space. Frenet formulas, the isoperimetric inequality, local theory of surfaces in Euclidean space, first and second fundamental forms. Gaussian and mean curvature, isometries, geodesics, parallelism, the Gauss-Bonnet Theorem.
    
    \textbf{Book}: Elementary Differential Geometry by Andrew Pressley
\end{enumerate}

\end{document}