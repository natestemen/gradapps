\documentclass[11pt]{article}

\usepackage[utf8]{inputenc}
\usepackage[margin=1.25in]{geometry}
\usepackage{microtype}
\usepackage{fancyhdr}
\usepackage{palatino}
\usepackage{tipa}
\usepackage{booktabs}
\usepackage{enumitem}
 
\pagestyle{fancy}
\fancyhf{}
\lhead{Nathaniel T. Stemen}
\rhead{Vrije Universiteit Amsterdam}

\lfoot{Course Information}
\rfoot{Page \thepage}

\begin{document}

\section*{Calculus/Analysis and Geometry}

\begin{center}
\begin{tabular}{lcc}
Course Name & ECTS Credits & Grade Received \\ \toprule
Calculus I & 8 & A \\
Calculus II & 8 & A \\
Calculus III: Multi Dimensional Calculus & 8 & A \\
Honors Analysis I & 8 & B+ \\
Reading in Mathematics I (Differential Geometry) & 6 & A
\end{tabular}
\end{center}

\subsubsection*{Course Objectives \& Textbook}
\begin{enumerate}
    \item Calculus I
    
    Limits, derivatives of functions defined by graphs, tables and formulas, differentiation rules for power, polynomial, exponential and logarithmic functions, derivatives of trigonometric functions, the product and quotient rules, the chain rule, applications of the chain rule, maxima and minima, optimization.The definite integral, the Fundamental Theorem of Calculus and interpretations, theorems about definite integrals, anti-derivatives.
    
    \textbf{Book}: Calculus by J.\ Stewart
    
    \item Calculus II
    
    his course covers techniques of integration, introduction to ordinary differential equations, improper integrals, numerical methods of integration, applications of integration, sequences, series, power series, approximations of functions via Taylor polynomials, Taylor series, functions of two variables, graphs of functions of two variables, contour diagrams, linear functions, functions of three variables.
    
    \textbf{Book}: Calculus by J.\ Stewart
    
    \item Calculus III: Multi Dimensional Calculus
    
    Vectors in the plane and space. Partial derivatives with applications, especially Lagrange multipliers. Double and triple integrals. Spherical and cylindrical coordinates. Surface and line integrals. Divergence, gradient, and curl. Theorems of Gauss and Stokes.
    
    \textbf{Book}: Calculus by J.\ Stewart
    
    \item Honors Analysis I
    
    This is an introduction to the rigorous treatment of the foundations of real analysis in one variable. It is based entirely on proofs. Students are expected to know what a mathematical proof is and are also expected to be able to read a proof before taking this class. Topics include: properties of the real number system, sequences, continuous functions, topology of the real line, compactness, derivatives, the Riemann integral, sequences of functions, uniform convergence, infinite series and Fourier series. Additional topics may include: Lebesgue measure and integral on the real line, metric spaces, and analysis on metric spaces.
    
    \textbf{Book}: Foundations of Mathematical Analysis by Richard Johnsonbaugh and W.\ E.\ Pfaffenberger

    \item Differential Geometry
    
    The geometry of curves and surfaces in Euclidean space. Frenet formulas, the isoperimetric inequality, local theory of surfaces in Euclidean space, first and second fundamental forms. Gaussian and mean curvature, isometries, geodesics, parallelism, the Gauss-Bonnet Theorem.
    
    \textbf{Book}: Elementary Differential Geometry by Andrew Pressley
\end{enumerate}

\section*{Linear algebra/Algebra}

\begin{center}
\begin{tabular}{lcc}
Course Name & ECTS Credits & Grade Received \\ \toprule
Linear Algebra \& Differential Equations & 8 & A \\
Advanced Linear Algebra \& Complex Variables & 6 & A \\
Linear Algebra I & 6 & A \\
Linear Algebra II & 6 & A
\end{tabular}
\end{center}

\subsubsection*{Course Objectives \& Textbook}
\begin{enumerate}
    \item Linear Algebra \& Differential Equations
    
    This course introduces vector concepts. Linear transformations. Matrices and Determinants. Characteristic roots and eigenfunctions.
    
    \textbf{Book}: Linear Algebra and its Applications by David C.\ Lay
    
    \item Advanced Linear Algebra \& Complex Variables
    
    The Gram-Schmidt process, inner product spaces and applications, singular value decomposition, LU decomposition, derivatives and Cauchy-Riemann equations, integrals and Cauchy integral theorem, power and Laurent series, residue theory.
    
    \textbf{Book}: Visual Complex Analysis by Tristan Needham and professor's notes.
    
    \item Linear Algebra I (graduate course)
    
    This course covers basic ideas of linear algebra: Groups, rings, fields, vector spaces, basis, dependence, independence, dimension. Relation to solving systems of linear equations and matrices. Homomorphisms, duality, inner products, adjoints and similarity.
    
    \textbf{Book}: Professor's notes, but studied with Abstract Algebra by 	David S. Dummit and Richard M. Foote.
    
    \item Linear Algebra II (graduate course)
    
    Topics covered are basic concepts of linear algebra continuing with: range, nullity, determinants and eigenvalues of matrices and linear homomorphisms, the polar decomposition and spectral properties of linear maps, orthogonality, adjointness and its applications.
    
    \textbf{Book}: Professor's notes and Finite Dimensional Vector Spaces by Paul Halmos.
\end{enumerate}

\section*{Differential equations}

\begin{center}
\begin{tabular}{lcc}
Course Name & ECTS Credits & Grade Received \\ \toprule
Linear Algebra \& Differential Equations & 8 & A \\
Applied Partial Differential Equations & 6 & A- \\
Mathematical Physics & 6 & A
\end{tabular}
\end{center}
\subsubsection*{Course Objectives \& Textbook}
\begin{enumerate}
    \item Linear Algebra \& Differential Equations
    
    Topics to be covered include: first-order equations including integrating factors; second-order equations including variation of parameters; series solutions; elementary numerical methods including Euler’s methods, Runge-Kutta methods, and error analysis; Laplace transforms; systems of linear equations; boundary-value problems.
    
    \textbf{Book}: A First Course in Differential Equations by Dennis Zill

    \item Applied Partial Differential Equations
    
    This course gives an overview of PDEs that occur commonly in the physical sciences with applications in heat flow, wave propagation, and fluid flow. Analytical as well as some numerical solution techniques will be covered, with a focus on applications.
    
    \textbf{Book}: Applied Partial Differential Equations by J.\ David Logan

    \item Mathematical Physics
    
    Mathematical preparation for further study in physics. Review of vectors and matrices, differential and integral operations, curvilinear coordinate systems, introduction to tensors, Dirac delta-function, complex variables, ordinary and partial differential equations, solutions to Laplace's equation.
    
    \textbf{Book}: Mathematical Methods in the Physical Sciences by Mary L.\ Boas
\end{enumerate}


\section*{Numerical analysis, scientific computing}

\begin{center}
\begin{tabular}{lcc}
Course Name & ECTS Credits & Grade Received \\ \toprule
Computational Physics & 8 & B- \\
Thesis for Bachelor of Science Degree & 6 & A
\end{tabular}
\end{center}

\subsubsection*{Course Objectives \& Textbook}
\begin{enumerate}
    \item Computational Physics

    An introduction to numerical methods. Solving ordinary differential equations, root finding, Fourier transforms, numerical integration, linear systems. Techniques are applied to projectile motion, oscillatory motion, planetary motion, potentials and fields, waves and quantum mechanics
    
    \textbf{Book}: Computational Physics by Nicholas Giordano and Hisao Nakanishi
    
    \item Thesis for Bachelor of Science Degree
    
    While working on my thesis I learned about topics ranging from simple root-finding procedures to numerically solving differential equations and extensively used the finite element method.
\end{enumerate}

\section*{Computer programming}

\begin{center}
\begin{tabular}{lcc}
Course Name & ECTS Credits & Grade Received \\ \toprule
Introduction to Programming \& Problem Solving & 8 & A- \\
Data Structures \& Algorithms & 8 & B
\end{tabular}
\end{center}

\subsubsection*{Course Objectives \& Textbook}
\begin{enumerate}
    \item Introduction to Programming \& Problem Solving

    The course covers fundamentals of computer programming and its underlying principles using the Python programming language.
    
    \item Data Structures \& Algorithms
    
    This course covers abstract data types and the implementation and use of standard data structures along with fundamental algorithms and the basics of algorithm analysis.
    
    \textbf{Book}: Introduction to Algorithms by T.\ H.\ Cormen, C.\ Stein, R.\ L.\ Rivest and C.\ E.\ Leiserson

\end{enumerate}

\section*{Probability}

\begin{center}
\begin{tabular}{lcc}
Course Name & ECTS Credits & Grade Received \\ \toprule
Introduction to Probability I & 4 & A \\
Probability II & 4 & A
\end{tabular}
\end{center}

\subsubsection*{Course Objectives}
\begin{enumerate}
    \item Introduction to Probability I \& Probability II
     
    Together, the courses cover axioms of mathematical probability, combinatorial analysis, binomial distribution, Poisson and normal approximation, random variables and probability distributions, generating functions, the Central Limit Theorem and Laws of Large Numbers, Markov Chains, and basic stochastic processes.
\end{enumerate}

\section*{Other}

\begin{center}
\begin{tabular}{lcc}
Course Name & ECTS Credits & Grade Received \\ \toprule
The Art of Mathematics & 4 & A- \\
Symbolic Logic & 8 & A \\
Topology & 8 & B- \\
Reading Seminar in Mathematics II (Graph Theory) & 8 & A \\
Thesis for Bachelor of Science Degree & 6 & A
\end{tabular}
\end{center}

\subsubsection*{Course Objectives \& Textbook}
\begin{enumerate}
    \item The Art of Mathematics
    
    This is an introductory course about Mathematics. Areas of Mathematics. History of Mathematics. Mathematical Methods. Great Mathematicians. Famous open and solved mathematical problems. The study of Mathematics. Mathematical Software.
    
    \textbf{Book}: How to Solve it by G.\ Polya

    \item Symbolic Logic
    
    This course introduces the methods and applications of propositional logic and relational predicate logic. The course looks at the concept of a formal language and covers semantic and proof-theoretic methods of testing arguments for validity. Semantic concepts of tautology, logical equivalence and consistency are compared with their proof-theoretic counterparts, and the notions of soundness and completeness of proof-theoretic methods are introduced.
    
    \textbf{Book}: Elementary Symbolic Logic by  W.\ Gustason and D.\ E.\ Ulrich
    
    \item Topology
    
    Set-theoretic preliminaries. Metric spaces, topological spaces, compactness, connectedness, covering spaces, and homotopy groups.
    
    \textbf{Book}: Topology by J.\ R.\ Munkres and Applied Point-Set Topology by E.\ A.\ Ok
    
    \item Graph Theory
    
    Some of the topics we will cover include: Matchings, cuts, flows, connectivity, planar graphs, graph colorings, random graphs, extremal graph theory, Ramsey theory, linear algebra methods, and expander graphs.
    
    \textbf{Book}: Professor's notes
    
    \item Thesis
    
    \textbf{Book}: Principles of Functional Analysis by M.\ Schechter and An introduction to Quantum Field Theory by M.\ E.\ Peskin and D.\ V.\ Schroeder
\end{enumerate}

\section*{Advanced Physics}

\begin{center}
\begin{tabular}{lcc}
Course Name & ECTS Credits & Grade Received \\ \toprule
Quantum Mechanics I & 6 & A \\
Quantum Mechanics II & 6 & A- \\
General Relativity & 8 & B-
\end{tabular}
\end{center}

\subsubsection*{Course Objectives \& Textbook}
\begin{enumerate}
    \item Quantum Mechanics I
    
    Topics include foundational experiments, wave-particle duality, wave functions, the uncertainty principle, the time-independent Schrödinger equation and its applications to one-dimensional problems and the hydrogen atom, angular momentum, and spin; Hilbert Space, operators, and observables; timeindependent perturbation theory; atomic spectra.
    
    \textbf{Book}: Quantum Mechanics by D.\ J.\ Griffiths

    \item Quantum Mechanics II
    
    Topics include the time-dependent Schrödinger equation, the Schrödinger and Heisenberg description of quantum systems, timedependent perturbation theory, scattering theory, quantum statistics, and applications to atomic, molecular, nuclear, and elementary particle physics.
    
    \textbf{Book}: Quantum Mechanics by D.\ J.\ Griffiths
    
    \item General Relativity
    
    Provides an introduction to general relativity, stressing physical phenomena and their connection to experiments and observations. Topics include special relativity, gravity as geometry, black holes, gravitational waves, cosmology, Einstein equations.
    
    \textbf{Book}: Gravity by J.\ B.\ Hartle
\end{enumerate}


\end{document}