\documentclass[11pt]{article}

\usepackage[utf8]{inputenc}
\usepackage[margin=1in]{geometry}
\usepackage{microtype}
\usepackage{fancyhdr}
\usepackage{palatino}
 
\pagestyle{fancy}
\fancyhf{}
\lhead{Nathaniel T. Stemen}
\rhead{Vrije Universiteit Amsterdam}

\lfoot{Motivation Letter (VUFP Scholarship)}

\renewcommand{\baselinestretch}{1.5}

\begin{document}

I am applying to the Masters program in Computer Science at Vrije Universiteit Amsterdam because of the faculties expertise in formal verification and automated theorem proving. Having spent the better part of the past three years as a professional software developer, I've seen where some of the benefits of formal verification could come into play. Formally verifying code before being shipped to users would be a major step for making applications more usable, and secure for all parties. Having also spent my bachelors degree studying mathematics, the benefits of formally verifying proofs, and trying to get the computer to do some itself is a clear benefit not only for proving theorems but also for building mathematical search engines. Given the reliance of our society on software, the applications of work in this field has the potential to be enormous, and benefit everyone who uses a computer. With this fields massive potential impact, and VU Amsterdam's expertise in formal verification, I believe this to be an ideal institution to obtain a Masters from.

One year into my bachelors degree, I knew I wanted to be mathematician who both taught courses and did original research. I was learning so much in my classes, and as I got a taste of research, I felt the thrill of discovering for the first time. Once I had that feeling I couldn't shake it. After graduating though, I chose to take a job to alleviate some financial pressures I had from my undergraduate degree. Even though this wasn't my first choice at the time, the skills I've developed while working will only benefit me as a graduate student, and I've even gotten to explore new areas of mathematics. Now I am prepared to go back to school to pursue my career as a mathematician, with the goal of eventually doing a PhD and one day pursuing research and teaching at the university level. With VU Amsterdam's FCC track, I will be able to place my future career on a solid mathematical foundation while also being exposed to a broad range of new ideas.

With the VUFP scholarship I will be able to devote more time to learning my proposed field, as well as diving into research. With the lessened financial burden that comes with a scholarship, I will be able to focus more time on being a student, and growing as a researcher, and spend less time making money to afford Amsterdam rent.

\end{document}