\documentclass[11pt]{article}

\usepackage[utf8]{inputenc}
\usepackage[margin=1in]{geometry}
\usepackage{microtype}
\usepackage{fancyhdr}
\usepackage{palatino}
\usepackage[shortlabels]{enumitem}
 
\pagestyle{fancy}
\fancyhf{}
\lhead{Nathaniel T. Stemen}
\rhead{University of Waterloo}

\lfoot{Supplementary Information}
\rfoot{Page \thepage}

\begin{document}

Provide a statement (maximum 500 word) addressing the following points:
\begin{enumerate}[a)]
    \item The area(s) of research you wish to pursue and why
    \item Your previous research experience if any
    \item Any previous coursework in quantum information, if any
\end{enumerate}

I am applying to the University of Waterloo's Quantum Information Masters program for it's expertise in both quantum information/computation and mathematics. Having completed my bachelors degree in both mathematics and physics, quantum information is one of the fields that takes some of the most interesting topics from both fields and weaves them into something entirely new. While quantum computers slowly gain power and ability, I believe it is important that we fully understand the theoretical foundations to this field. This has led me mostly to the work of professors William Slofstra and Vern Paulsen. In particular Slofstra's work on multi-prover proofs and Paulsen's work on embezzlement are topics I would love to learn more about. While quantum information weaves together mathematics and physics, it also brings together many aspects from computer science which I've become more fascinated with during my career as a software developer. With branches of computational complexity devoted to quantum algorithms, it is important that researchers in this field have experience in all three domains. My combined mathematics and physics degree, along with my experience as a professional software developer suite me well for this field.

During my undergraduate studies, I had the opportunity to spend two summers working at Yale University where I worked on an experiment searching for a hypothetical sterile neutrino. There my focus was on building simulations of detectors, and measuring properties of various liquid scintillators, but I also explored the mathematics behind neutrino oscillation where I first learned about operator algebras and spectra. I also got involved in the NYU mathematics department working on optical vortex solitons which turned into my Bachelor's thesis on Q-Balls. This research was mainly focused within functional analysis, but in order to write my thesis, I had to understand lots of classical and quantum field theory to give physics explanations as well as mathematical ones for the results we were seeing. This is when I started believing the field of mathematical physics was perfect for my research interests. It allowed my to learn and study pure mathematics while finding applications and many problems within physics (and sometimes even computer science) to apply the knowledge to.

While I have not taken any formal courses in quantum information/computation, I have taken two graduate quantum mechanics classes, and sat in on a graduate mathematical physics course which focused on field theories. This has given me a strong foundation of both the mathematics, and physics that underlay much of quantum information. This knowledge, along with what I've learned through research prepare me well for this masters program. Along with the University of Waterloo's Institute for Quantum Computing, I believe this will be a great place to pursue the interdisciplinary research I hope to engage in.


\end{document}