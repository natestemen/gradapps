\documentclass[11pt]{article}

\usepackage[utf8]{inputenc}
\usepackage[margin=1in]{geometry}
\usepackage{microtype}
\usepackage{fancyhdr}
\usepackage{palatino}
\usepackage{csquotes}
\usepackage{titlesec}

\titlespacing*{\section}{0pt}{1.5ex}{1ex}
\titlespacing*{\subsubsection}{0pt}{1ex}{.5ex}
 
\pagestyle{fancy}
\fancyhf{}
\lhead{Nathaniel T. Stemen}
\rhead{KTH Royal Institute of Technology \& Stockholm University}

\lfoot{Entry Requirements}
\rfoot{Page \thepage}

\renewcommand{\baselinestretch}{1.17}

\begin{document}

In this document I explain how I meet the entry requirements for the joint Masters of Mathematics coordinated by KTH and Stockholm University.

\section*{Abstract Algebra/Groups and Rings}

During my last year at university I took both Graduate Linear Algebra I \& II. The course descriptions as written by the professor are as follows.
\subsubsection*{Graduate Linear Algebra I (MA-GY 7033)}
\begin{displayquote}
This course covers basic ideas of linear algebra: Groups, rings, fields, vector spaces, basis, dependence, independence, dimension. Relation to solving systems of linear equations and matrices. Homomorphisms, duality, inner products, adjoints and similarity.
\end{displayquote}

\subsubsection*{Graduate Linear Algebra II (MA-GY 7043)}
\begin{displayquote}
This course continues MA-GY 7033. Topics covered are basic concepts of linear algebra continuing with: range, nullity, determinants and eigenvalues of matrices and linear homomorphisms, the polar decomposition and spectral properties of linear maps, orthogonality, adjointness and its applications.
\end{displayquote}

While these courses were not explicitly named "Abstract Algebra" we covered all the necessary predecessors in order to rigorously define a vector space and I believe these courses cover the requisite abstract algebra.

\section*{Foundations of Mathematical Analysis}

I took two analysis focused courses during my bachelors studies.

\subsubsection*{Honors Analysis I}
In this course we used ``Foundations of Mathematical Analysis'' by Richard Johnsonbaugh and W.E. Pfaffenberger and the course description follows.
\begin{displayquote}
This is an introduction to the rigorous treatment of the foundations of real analysis in one variable. It is based entirely on proofs. Students are expected to know what a mathematical proof is and are also expected to be able to read a proof before taking this class. Topics include: properties of the real number system, sequences, continuous functions, topology of the real line, compactness, derivatives, the Riemann integral, sequences of functions, uniform convergence, infinite series and Fourier series. Additional topics may include: Lebesgue measure and integral on the real line, metric spaces, and analysis on metric spaces.
\end{displayquote}

\subsubsection*{Advanced Linear Algebra and Complex Variables}
\begin{displayquote}
Topics covered include: The Gram-Schmidt Process, inner product spaces and applications, singular value decomposition, LU decomposition, derivatives and Cauchy-Riemann equations, integrals and Cauchy integral theorem. Power and Laurent Series, residue theory. 
\end{displayquote}

\end{document}