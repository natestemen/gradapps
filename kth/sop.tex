\documentclass[11pt]{article}

\usepackage[utf8]{inputenc}
\usepackage[margin=1in]{geometry}
\usepackage{microtype}
\usepackage{fancyhdr}
\usepackage{palatino}
 
\pagestyle{fancy}
\fancyhf{}
\lhead{Nathaniel T. Stemen}
\rhead{KTH Royal Institute of Technology \& Stockholm University}

\rfoot{Letter of Intent}
% \rfoot{Page \thepage}

\renewcommand{\baselinestretch}{1.23}

\begin{document}

\noindent
To Whom It May Concern,

I am applying for the Master's program in Mathematics run jointly by the KTH Royal Institute of Technology and Stockholm University. After receiving my Bachelors degree in 2017 I took a job as a professional software developer. Initially I was worried that taking time off to work might not be attractive to some graduate schools, but I believe it has only made my passion for mathematics stronger, and my desire to attend graduate school more intense. Having worked for two years, I've been able to explore some of the mathematical landscape to find what interests me most. Having completed my Bachelors degree in mathematics and physics, found work in a computer science adjacent field, I discovered my interest lies most in problems which involve multiple fields of mathematics, or mathematics adjacent fields. For example, in my thesis I thoroughly enjoyed learning some of the requisite classical and quantum field theories that underlay our work. This allowed me to view our research from new angles, and give better explanations for results we were seeing. For this reason I have fell in love with many fields that take ideas and tools from advanced mathematics, and mix it with ideas from other fields.

I have chosen to apply to this program because it would allow me to further develop a strong mathematical foundation with the plethora of courses offered jointly between the two universities, while also giving me an opportunity to further explore my mathematical interests. Considering that I have been out of formal academia for two years time I think it is appropriate that I use a masters to further sharpen my abilities, and with this program's basic block, followed by the profile and broadening block, I think it will do just that. With the large amount of researchers between the two universities, there are many professors I would be honored to be able to work with to write a masters thesis. For example Professor P\"{a}r Kurlberg's work on number theory and quantum chaos would allow me to use my background studying mathematical physics, and explore the fascinating world of number theory. Kathl\'{e}n Kohn's work on applying algebraic geometry to study growing fields like machine learning, computer vision and perhaps most interestingly cryptography would also be projects I would love to learn more about (and perhaps apply some of my programming knowledge too). Lastly, Johan Håstad's work on computational complexity has fascinated me as we often discuss basic problems in this field in software, but learning more about the mathematical underpinnings would allow me to better understand these problems.

As stated in the SMC's ``Aims and Objectives", this program is meant to help lay the foundation for courses at the advanced and graduate level, as well as build a communication platform for developing mathematicians. With future goals to pursue a PhD in mathematics, I trust this program will further develop my mathematical foundations, build my problem solving abilities, and help me advance my skills as a researcher. Along with the many educational initiatives the SMC runs, I would relish in the opportunity to attend any of the Algebra \& Geometry, Logic, and Theoretical Computer science seminars. With the plethora of resources available between the two departments, I believe this to be an excellent program to continue my mathematical career.

\noindent
Best,

Nathaniel T. Stemen
\end{document}