\documentclass[11pt]{article}

\usepackage[utf8]{inputenc}
\usepackage[margin=1in]{geometry}
\usepackage{microtype}
\usepackage{fancyhdr}
\usepackage{palatino}
 
\pagestyle{fancy}
\fancyhf{}
\lhead{Nathaniel T. Stemen}
\rhead{University of Hawai'i at Mānoa}

\lfoot{Statement of Objective}
\rfoot{Page \thepage}

\renewcommand{\baselinestretch}{1.38}

\begin{document}

Tricks! ``Tricks'' have always inspired my passion and relentless drive for solutions and mastery. Before college, I devoted my free time to conquering the physical skills and knowledge required for high-level skateboard tricks. Silly? Not if you consider that I was unwittingly training myself in keen observation, physical dynamics, intense mental focus, hard work on fundamentals, imaginative application of available tools, teamwork, and personal initiative. Toward the end of high school and at NYU, my interests in seemingly mysterious physical phenomena evolved into an immersion in mathematics.

My passion for mathematics deepened during my research at Yale University where we searched for a hypothetical sterile neutrino. There I optimized our data collection pipeline using pulse shape discrimination and built a simulation of the detector that incorporated realistic photomultiplier tubes and quantum efficiency curves using C++. Using my simulation we were able to design a light-guide that collected 30\% more photons and increased detector uniformity. In between writing code and taking samples in the lab, I began exploring the mathematics behind neutrino oscillation which led me to the world of linear algebra. I learned about mixing matrices and angles, but also more foundational material about unitary transformations and operator spectra. As I wrapped up my time at Yale, I presented a poster at a conference and published two papers with with the collaboration.

In my third year, I took a reading course on differential geometry which was solely discussion based. Embracing this format, I was stunned how easily direct, imaginative, and new discovery became. I learned about classical differential geometry, and even built a cardboard and string model of a hyperboloid to demonstrate how it was doubly-ruled. This course taught me mathematics doesn't always have to be pen and paper, and how enjoyable it can be when working on applied problems. Bettering my presentation skills through weekly reports to my peers plays into my ultimate goal of collaborating on research teams and teaching at the university level.

Working with Prof. Luciano Medina on optical vortex solitons, our mission was the prove the existence of solutions for a class of nonlinear Schr\"{o}dinger equations. Using a variational method, we reframed the problem in the language of functional analysis and operator algebras. Doing this allowed us to use powerful tools from functional analysis and apply them to a functional which we were trying to minimize. We were thus able to give necessary conditions for nontrivial solutions and prove the existence of solutions via constrained minimization. To further understand the problem, I computed solutions using python and verified the numerics from previous papers. While the work was highly specialized, it bolstered my fundamentals in functional analysis, and taught me lots about the world of numerical computing. This work led to a poster presentation for NYU faculty, and talk at Rice University.

Although my program did not require a thesis, I chose to write one on Q-Balls (a non-topological soliton) knowing it would help prepare me for future research. Studying the foundational classical and quantum field theories which underlie much of our work allowed me to give physical explanations as well as mathematical ones for our results. Through proving integral inequalities and functionals to be lower semi-continuous and coercive, we showed spherically symmetric Q-Ball solutions existed. With this work, I continued developing my numerical computing skills, and learned that I enjoy research that engages both pure and applied mathematics.

This has largely drawn me to fields that apply advanced mathematical and computational techniques to solve problems within mathematics and adjacent fields. I loved how in my thesis I got to combine mathematics from functional analysis, group theory, numerical computing and lots of physics. Getting all these areas to ``play nice'' with each other, and come together in harmony to help solve a problem felt so rewarding. For this reason applied mathematics is an ideal fit for me, and the University of Hawaii's research in this field align well with my interests. Professor Yuriy Mileyko's work on topological data analysis has peaked my interest partially because it applied advanced techniques from topology, but also because my experience with data science projects allows me to view these problems with more than just a theoretical eye.

After graduating from NYU I opted to take a job as a software developer/data scientist at the academic writing enterprise; Overleaf. Having no professional software experience, I was nervous that I was in over my head. However, I demonstrated that I could pick up necessary skills on the job. My time is split between working on our web application and handling data science related tickets, but even do \LaTeX{} support.

Overleaf has also sent me to a multitude of mathematics conferences to represent our company. While I am at conferences to interact with users, I have the freedom to attend talks and lectures as I wish. This has allowed me to further explore fields I'm interested in and have attended talks ranging from linguistic diversity in the classroom, to applied category theory. All in all becoming a more proficient programmer has only strengthened my mathematical ability. Running simulations of shuffling cards, building computational models of dual numbers, and learning about new numerical computing languages like Julia are just a few examples of how programming has helped deepen and broaden my mathematical knowledge. Maintaining this connection has allowed me to grow my enthusiasm for mathematics, and develop a more specialized interest in scientific, and numerical computing.

My treasured moments (in skateboarding and research) have resulted from discipline, struggle, and trust. Through the University of Hawai'i's Ph.D.\ in mathematics, I will be prepared to continue my career asking difficult questions in applied mathematics. With the departments variety of graduate students from different backgrounds, I trust this program will help me succeed in my mathematical pursuit. And I may skateboard around Honolulu as well.

\end{document}